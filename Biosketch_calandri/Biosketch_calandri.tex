%% For Use With NIH Biosketch LaTeX Class
%% Added to RMarkdown by Tyson S. Barrett, PhD

% Document class and options
\documentclass{nihbiosketch}
\usepackage{natbib}
\usepackage{bibentry}
\bibliographystyle{apalike}
\providecommand{\tightlist}{\setlength{\itemsep}{0pt}\setlength{\parskip}{0pt}}

% Pandoc header




\name{Ismael Luis Calandri}
\eracommons{}
\position{Asistant professor}


\begin{document}



  \nobibliography{papers.bib}


  \begin{education}
      Universidad Catolica de Cordoba & MD & 01/2010 & Medicine \\
      Fleni & Neurologist & 05/2015 & Neurology residence \\
      Fleni & Behavioral neurologist & 05/2017 & Fellowship in cognitive
neurology \\
      Fleni & - & 05/2016 & Fellowship in cognitive neuroimaging \\
      Universidad de Buenos Aires & MS & 09/2017 & Statistics for health
science \\
    \end{education}



\hypertarget{personal-statement}{%
\section*{Personal Statement}\label{personal-statement}}
\addcontentsline{toc}{section}{Personal Statement}

As a neurologist and dementia researcher, I integrate clinical expertise
with statistical methods to study Alzheimer's disease (AD) progression.
My work focuses on neuroimaging\textsuperscript{1,2}, cognitive
trajectories, and environmental risk factors\textsuperscript{3,4},
aiming to improve diagnostics and patient care, especially in vulnerable
populations.

My research leverages multidimensional data to explore AD heterogeneity,
particularly in low- and middle-income countries. I contribute to
neuroimaging-clinical integration projects, advancing predictive models
and intervention strategies. My primary interest is the development of
diagnostic tools for early interventions in Alzheimer's disease in Latin
American countries, combining clinical insights with statistical
approaches to enhance dementia prevention efforts.

Additionally, I collaborate with the Laboratorio del Tiempo, led by Drs.
Casiraghi and Spiousas, where I have contributed to the analysis of
actimetry and lunar cycles in healthy subjects\textsuperscript{5}. This
work expands my expertise in circadian rhythms and environmental
influences on human behavior, reinforcing my multidisciplinary approach
to neuroscience research.

\begin{enumerate}
  \item \bibentry{German2021im}
  \item \bibentry{Cubas_Guillen2024mt}
  \item \bibentry{Calandri2024um}
  \item \bibentry{Calandri2020}
  \item \bibentry{Casiraghi2021-qs}
\end{enumerate}

\hypertarget{positions-and-honors}{%
\section*{Positions and Honors}\label{positions-and-honors}}
\addcontentsline{toc}{section}{Positions and Honors}

\hypertarget{positions-and-employment}{%
\subsection*{Positions and Employment}\label{positions-and-employment}}
\addcontentsline{toc}{subsection}{Positions and Employment}

\begin{datetbl}
2024-- & Associate Research Professor,Universidad de San Andres, Buenos Aires, Argentina \\
2017-- & Assistant professor on statistics and research methodology, Universidad Maimonides, Buenos Aires, Argentina \\
2015-- & Associate Professor of Clinical Neurology, Universidad de Buenos Aires, Buenos Aires, Argentina \\
2015--2022 & Assistant Professor of Aphasiology and neuropsychology, Universidad del Salvador, Buenos Aires, Argentina \\
\end{datetbl}

\hypertarget{other-experience-and-professional-memberships}{%
\subsection*{Other Experience and Professional
Memberships}\label{other-experience-and-professional-memberships}}
\addcontentsline{toc}{subsection}{Other Experience and Professional
Memberships}

\begin{datetbl}
2018-- & Statistician leader, Fleni, Buenos Aires, Argentina \\
2018-- & LatAm-FINGERS,Executive committee member, Buenos Aires, Argentina \\
2023-- & iLEADS-Argentina, coPI \\
2015-- & ADNI-Argentina, subPI \\
\end{datetbl}

\hypertarget{honors}{%
\subsection*{Honors}\label{honors}}
\addcontentsline{toc}{subsection}{Honors}

\begin{datetbl}
2024 & Prof Dr J M Ramos Mejia Research Award, Neurological Society of Argentina \\
2017 & Prof Dr J M Ramos Mejia Research Award, Neurological Society of Argentina \\
2014 & Sociedad Neurologica Argentina Best Research Award, Neurological Society of Argentina \\
2010 & Summa cum laude, Universidad catolica de Cordoba, Argentina \\
\end{datetbl}

\hypertarget{contribution-to-science}{%
\section*{Contribution to Science}\label{contribution-to-science}}
\addcontentsline{toc}{section}{Contribution to Science}

\begin{enumerate}

\item My line of work has focused on identifying clinical markers of typical and atypical forms of Alzheimer's disease in Latin America in collaboration with global networks.

\begin{enumerate}
  \item \bibentry{Calandri2023-yf}
  \item \bibentry{Chapleau2024-eo}
  \item \bibentry{Bergeron2018-il}
  \item \bibentry{Pomilio2022-xz}
  \item \bibentry{Mendez2018-io}
\end{enumerate}


\item My other line of work has primarily focused on identifying risk factors and developing non-pharmacological strategies for early intervention.

\begin{enumerate}
  \item \bibentry{Kivipelto2020-nl}
  \item \bibentry{Crivelli2023-el}
  \item \bibentry{Calandri2024u}
  \item \bibentry{Paradela2024-dq}
  \item \bibentry{Calandri2020-my}
\end{enumerate}

\end{enumerate}



  \subsection*{Complete List of Published Work in MyBibliography:} 
  \url{\url{https://scholar.google.com/citations?hl=es\&view_op=list_works\&gmla=AOv-ny_7qIAnldsTzuKwnpX8WXHUf1oLfDa6e-xHtsRdrL3SmDjHqhtQtL3jAjTqOTuLPR-jHaBFjiBWrxSprGbYJXE3F3dKxGqiLPzsTPB-rUI\&user=6W7W2D0AAAAJ}}
 

  \section{Research Support}
  
        \subsection*{Ongoing Research Support}
    
            
      \grantinfo{24AARG-D-1246942}{Custodio, Nilton - Calandri,
Ismael}{07/2024 -}
      {Resilience in Andean American Indigenous with Alzheimer's
disease}
      {Evaluate the impact of risk factors on brain resilience to
Alzheimer's disease using neuroimaging within a Mendelian randomization
model.}
      {Role: coPI}
      
      \bigskip
            
      \grantinfo{SG-22-1009956-iLEADS}{Calandri, Ismael - Ricardo
Allegri}{9/1/2022 - 12/31/2023}
      {ALZ Strategic Initiatives: InternationalLongitudinal Early-onset
AD Study}
      {Study sporadic forms of early-onset dementia in presenile
subjects through longitudinal follow-up.}
      {Role: PI}
      
      \bigskip
          
      
  
    \subsection*{Completed Research Support}
  
          
      \grantinfo{PICTO-COVID-SECUELAS-00016}{Allegri Ricardo}{Over}
      {Neurocognitive sequelae of SARS-CoV-2: A multidimensional
clinical-biological approach.}
      {Clinical characterization of subjects with post-COVID
neurocognitive symptoms through imaging and plasma biomarkers.}
      {Role: coPI}
      
      \bigskip
      
  



\end{document}
